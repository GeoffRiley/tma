\documentclass[a4paper,12pt]{article}

\usepackage[pdfbookmark,roman,cleveref,legacy,boing]{tma}

\myname{Peter McFarlane}
\mypin{A1234567}
\mycourse{\LaTeX 101}
\mytma{sample}

%\marginnotes

\begin{document}
\maketitle

\hfill\fbox{Update for 2024/11/08 v1.12}

%% Notes for the latest update
\begin{center}
\begin{minipage}{0.9\linewidth}
\rule[0.4pt-1em]{0.4pt}{1em}\hrulefill\rule[0.4pt-1em]{0.4pt}{1em}\vskip 10pt

This document is provided as an example of how to use the \textbf{tma} package.

With the 8\nth\,November 2024 version of the package there has been a rewrite of
the majority of the package along with the renaming of a couple of the commands.
\verb|\C|, representing the set of complex numbers, has been renamed to \verb|\Complex|,
and \verb|\vec{}|, for setting traditional vectors, has been renamed to \verb|\vect{}|.
In both instances, the renaming has been done to avoid naming clashes from other packages.

To use the options start your document with:
\begin{verbatim}
\documentclass[a4paper,12pt]{article}
\usepackage[OPTION]{tma}

\myname{...
\end{verbatim}

Where \textbf{OPTION} is one of the following
\begin{description}
  \item[{[}roman{]}] Questions numbered as 1, 1(i), 1(i)(a)\dots
  \item[{[}alph{]}] \emph{Default} Questions numbered as 1, 1(a), 1(a)(i)\dots
\end{description}

The 26\nth\,October 2016 version of the package added two further options which allow extended referencing and the adding of
bookmarks for questions into pdf output.

\begin{description}
  \item[{[}cleveref{]}] Prepares the questions, qparts and subqparts to be referenced using the \verb|cleveref| package.
  \item[{[}pdfbookmark{]}] Automatically adds pdf bookmarks for each question, qpart and subqpart. This uses the \verb|hyperref| package.
\end{description}

The 8\nth\,November 2024 version added one further option.

\begin{description}
	\item[{[}legacy{]}] reactivate the old conflicting commands of \verb|\C| and \verb|\vec|. This is intended for use \textbf{only} when using the current package with an old document.
\end{description}

See the \verb|notes.pdf| file for further details of available options.

\vskip 10pt
\rule{0.4pt}{1em}\hrulefill\rule{0.4pt}{1em}

\end{minipage}
\end{center}
\pagebreak

\begin{question}
\qpart
We have $1=10^0$ and $1+2+3+4=10^1$. Prove that there are no other powers of ten which are the sum of the first $n$ integers.

We have:
\begin{gather*}
	\sum_{i=1}^n i=\frac{(n)(n+1)}{2}
\intertext{Let}
	\frac{(n)(n+1)}{2}=10^x\\
	\Rightarrow (n)(n+1)=2^{x+1}5^x
\end{gather*}

Now, either $n$ is odd, or $n+1$ is odd.

Consider the case where $n$ is odd:\\
By the Fundamental Theorem of Arithmetic, $n$ can only have the prime factors 2 or 5.  Since it is odd, it can only be a perfect power of 5. Now, $n+1$ also can only have the prime factors of 2 or 5.  If $n$ is divisible by 5, then $n+1$ is not divisible by 5.  Therefore $n+1$ is a perfect power of 2. Therefore:
\begin{gather*}
	n=5^x\qquad \text{and}\qquad n+1=2^{x+1}\\
	\Rightarrow x=0
	\intertext{(for any higher $x$, $5^x\gg 2^{x+1}$)}\\
	\Rightarrow n=1
\end{gather*}

Now consider the case where $n+1$ is odd:\\
By similar arguments to above, $n+1$ must be a perfect power of 5 and $n$ must be a perfect power of 2.
\begin{gather*}
	n=2^{x+1}\qquad \text{and}\qquad n+1=5^{x}\\
	\Rightarrow x=1
	\intertext{(for any higher $x$, $5^x\gg 2^{x+1}$)}\\
	\Rightarrow n=4
\end{gather*}

Therefore $n=1$ and $n=4$ are the only solutions to the original problem.  \hfill \qed\\[1cm]

\qpart[3]
\qsubpart
Show that:
\begin{gather*}
	 \sum_{x=1}^{n}x(x+1)=\frac{n(n+1)(n+2)}{3}\\
\end{gather*}


Let
\begin{gather*}
	f(n)=\frac{n(n+1)(n+2)}{3}\\
\end{gather*}

Now, adding the n+1 term to the above
\begin{gather}\notag
\begin{aligned}
	f(n)+(n+1)(n+2)&=\frac{n(n+1)(n+2)}{3}+(n+1)(n+2)\\
	&=\frac{\left(n^3+3n^2+2n+\right)}{3}+n^2+3n+2\\
	&=\frac{\left(n^3+6n^2+11n+6\right)}{3}\\
	&=\frac{\left((n+1)(n+2)(n+3)\right)}{3}\\
	&=f(n+1)
\end{aligned}
\end{gather}

Therefore, if $f(n)$ is valid, then so is $f(n+1)$.

Since $1\times2=2=\frac{1\times2\times3}{3}=f(1)$, then $f(n)$ is valid for all $n\geq1$. \hfill \qed\\[1cm]

\qsubpart
Show that:
\begin{gather*}
	 \sum_{x=1}^{n}x^4=\frac{n(n+1)(2n+1)(3n^2+3n-1)}{30}\\
\end{gather*}


Let
\begin{gather*}
	f(n)=\frac{n(n+1)(2n+1)(3n^2+3n-1)}{30}\\
\end{gather*}

Now, adding the n+1 term to the above
\begin{align}
	f(n)+(n+1)^4&=\frac{n(n+1)(2n+1)(3n^2+3n-1)}{30}+(n+1)^4\notag \\
	&=\frac{1}{30}(6n^5+15n^4+10n^3-n)+(n^4+4n^3+6n^2+4n+1)\notag \\
	&=\frac{1}{30}(6n^5+15n^4+10n^3-n+30n^4+120n^3+180n^2+120n+30)\notag \\
	&=\frac{1}{30}(6n^5+45n^4+130n^3+180n^2+119n+30)\label{a}
\end{align}

Now,
\begin{align}
	f(n+1)&=\frac{(n+1)\big((n+1)+1\big)\big(2(n+1)+1\big)\big(3(n+1)^2+3(n+1)-1\big)}{30}\notag \\
	&=\frac{1}{30}(n+1)(n+2)(2n+3)(3n^2+9n+5)\notag \\
	&=\frac{1}{30}(6n^5+45n^4+130n^3+180n^2+119n+30)\label{b}
\end{align}

Comparing equation~\eqref{a} with equation~\eqref{b} we see that
\begin{gather}\notag
f(n)+(n+1)^4=f(n+1)
\end{gather}

Therefore, if $f(n)$ is valid, then so is $f(n+1)$.

Since $1^4=1=\frac{1\times2\times3\times5}{30}=f(1)$, then $f(n)$ is valid for all $n\geq1$.\hfill \qed

\end{question}


\begin{question}[3]
Find the general solution of the equation
\begin{equation}
\label{e}
3\deriv{^2y}{x^2}+4\deriv{y}{x}+y=x^2
\end{equation}
The auxillary equation is
\begin{equation}
3\lambda^2+4\lambda+1=0
\end{equation}
which factorises to
\begin{equation}
(\lambda+1)(3\lambda+1)=0
\end{equation}
and so has solutions
\begin{equation}
\lambda=-1 \mbox{\ and\ }\lambda=-\frac{1}{3}
\end{equation}
As both roots are real and distinct, the complementary function is
\begin{equation}
\label{d}
y_c=C\e^{-x}+D\e^{-\frac{1}{3}x}
\end{equation}
Now, let us find the particular integral.  As the right hand side of equation \ref{e} is $x^2$, our trial solution is the polynomial
\begin{equation}
y_p=px^2+qx+r
\end{equation}
\begin{equation}
\Rightarrow \deriv{y_p}{x}=2px+q
\end{equation}
\begin{equation}
\Rightarrow \deriv{^2y_p}{x^2}=2p
\end{equation}
Substituting the trial particular integral into equation \ref{e}
\begin{equation}
6p+8px+4q+px^2+qx+r=x^2
\end{equation}
\begin{equation}
\Rightarrow px^2+(8p+q)x+(6p+4q+r)=x^2
\end{equation}
\begin{equation}
\Rightarrow p=1,\ \ q=-8,\ \ r=26
\end{equation}
Thus the particular integral is
\begin{equation}
\label{c}
y_p=x^2-8x+26
\end{equation}
and combining equation \ref{d} with equation \ref{c}, by the rule of superposition, we get the general solution of equation \ref{e} to be
\begin{equation}
y=C\e^{-x}+D\e^{-\frac{1}{3}x}+x^2-8x+26
\end{equation}
\end{question}
\end{document}

