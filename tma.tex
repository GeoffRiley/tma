%% This package may be freely used, especially by, but not limited to, students, lecturers, and
%% staff of the Open University. It has been created by the efforts of many who are now or
%% have been connected with the Open University Students Association. No acknowledgement is
%% _required_ for using this package within the production of a _Tutor Marked Assessment._
%%
%%
%% Adapted by Peter McFarlane from various sources.
%% All errors of style or content are mine or subsequent contributor.
%% Acknowledgements to Bob Margolis, and also Rob Lynas (from whom some macros are plagiarised).
%% Further contributions from Steve Mayer and Tim Dale.
%% My apologies for not annotating this file, I shall do so soon (yeah, right).
%% Annotations, in part, and further modification by Geoff Riley.
%%
%%
%% Package Options
%%    [alph]      (default) question numbering as 1(b)(iii)
%%    [roman]     varies question numbering to sequence used by M381 i.e. 1(ii)(c)
%%    [cleveref]  question numbering creates automatic referencing for use with cleveref package
%%    [pdfbookmark] add PDF bookmarks for each question using hyperref package
%%    [legacy]    enables old definitions of \vec and \C for backward compatibility
%%
%% To use a package option, place the option(s) before the package name:
%%    \usepackage[roman,cleveref,legacy]{tma}
%%
%% v1.12 2024/11/08
%%    Standardized package name to 'tma' to make it compatible with CTAN
%%    Avoided redefining standard LaTeX commands
%%    Consolidated geometry settings
%%    Adjusted loading order of packages
%%    Improved code readability and comments
%%    Added 'legacy' option to allow old definitions of \vec and \C
%%
\ProvidesPackage{tma}[2024/11/08 v1.12 TMA document style]
\NeedsTeXFormat{LaTeX2e}

%%%%%%%%%%%%%%%%%%%%%%%%%%
%% Package Initialization
%%%%%%%%%%%%%%%%%%%%%%%%%%

\newcommand{\name}{\relax}
\newcommand{\tma}{\relax}
\newcommand{\course}{\relax}
\newcommand{\pin}{\relax}

\title{\textbf{TMA: \course-\tma}}
\author{\textbf{\name\space\pin}}

\newcommand{\myname}[1]{\renewcommand{\name}{#1}}
\newcommand{\mytma}[1]{\renewcommand{\tma}{#1}}
\newcommand{\mycourse}[1]{\renewcommand{\course}{#1}}
\newcommand{\mypin}[1]{\renewcommand{\pin}{#1}}

%%%%%%%%%%%%%%%%%%%%%%%%%%%%
%% Package Loading
%%%%%%%%%%%%%%%%%%%%%%%%%%%%

\RequirePackage{amsmath}
\RequirePackage{amssymb}
\RequirePackage{amsthm}
\RequirePackage{wasysym}
\RequirePackage{bm}
\RequirePackage{upgreek}
\RequirePackage{graphicx}
\RequirePackage{lastpage}
\RequirePackage{xifthen}
\RequirePackage{verbatim}
\RequirePackage{fancyhdr}
\RequirePackage{geometry}

%%%%%%%%%%%%%%%%%%%%%%%%%%%%
%% Geometry Settings
%%%%%%%%%%%%%%%%%%%%%%%%%%%%

\geometry{
  headheight=10mm,
  headsep=5mm,
  bottom=25mm,
  footskip=15mm,
  left=30mm,
  right=30mm,
  marginparwidth=0mm,
  marginparsep=0mm,
  includemp
}

%%%%%%%%%%%%%%%%%%%%%%%%%%%%
%% Margin Notes
%%%%%%%%%%%%%%%%%%%%%%%%%%%%

\newcommand{\marginnote}[1]{\marginpar{#1}}
\newcommand{\marginnotes}{
  \geometry{
    marginparwidth=40mm,
    marginparsep=5mm,
    left=20mm,
    right=15mm
  }
}

%%%%%%%%%%%%%%%%%%%%%%%%%%%%
%% Question Numbering
%%%%%%%%%%%%%%%%%%%%%%%%%%%%

\newcounter{question}
\newcounter{qpart}[question]
\newcounter{qsubpart}[qpart]
\renewcommand{\thequestion}{\arabic{question}}

%%%%%%%%%%%%%%%%%%%%%%%%%%%%
%% Option Handling
%%%%%%%%%%%%%%%%%%%%%%%%%%%%
\typeout{Defining boolean flags.}
% Define boolean flags
\newif\iftma@roman
\newif\iftma@usecleveref
\newif\iftma@usepdfbookmark
\newif\iftma@legacy

\typeout{Setting default options.}
% Set default options
\tma@romanfalse          % Default numbering is 'alph'
\tma@useclevereffalse    % Default is not to use cleveref
\tma@usepdfbookmarkfalse % Default is not to use pdfbookmark
\tma@legacyfalse         % Default is not to use legacy definitions

\typeout{Defingin commands with default values.}
% Define commands with default values
\renewcommand{\theqpart}{\alph{qpart}}
\renewcommand{\theqsubpart}{\roman{qsubpart}}
\newcommand{\tma@crefname}[3]{\relax}
\newcommand{\tma@stepcounter}[1]{\stepcounter{#1}}
\newcommand{\tma@bookmark}[3][]{\relax}
\newcommand{\tma@pageref}[1]{\pageref{#1}}

\typeout{Declaring options.}
% Declare options
\DeclareOption{roman}{
	\tma@romantrue
}

\DeclareOption{alph}{
	\tma@romanfalse
}

\DeclareOption{cleveref}{
	\tma@useclevereftrue
}

\DeclareOption{pdfbookmark}{
	\tma@usepdfbookmarktrue
}

\DeclareOption{legacy}{
	\tma@legacytrue
}

\typeout{Processing options.}
% Process options
\ProcessOptions\relax

%%%%%%%%%%%%%%%%%%%%%%%%%%%%
%% Set Up Package Based on Options
%%%%%%%%%%%%%%%%%%%%%%%%%%%%

% Set question numbering
\iftma@roman
	\renewcommand{\theqpart}{\roman{qpart}}
	\renewcommand{\theqsubpart}{\alph{qsubpart}}
\else
	\renewcommand{\theqpart}{\alph{qpart}}
	\renewcommand{\theqsubpart}{\roman{qsubpart}}
\fi

% Load hyperref if necessary
\iftma@usepdfbookmark
	\RequirePackage[
		bookmarks=true,
		colorlinks=true,
		linkcolor=blue
	]{hyperref}
\fi

% Load cleveref if necessary
\iftma@usecleveref
	% Ensure hyperref is loaded before cleveref
	\@ifpackageloaded{hyperref}{}{%
		\RequirePackage{hyperref}
	}
	\RequirePackage{cleveref}
	% Redefine commands for cleveref
	\renewcommand{\tma@crefname}[3]{\crefname{#1}{#2}{#3}}
	\renewcommand{\tma@stepcounter}[1]{\refstepcounter{#1}}
\fi

% Redefine commands for pdfbookmark
\iftma@usepdfbookmark
	\renewcommand{\tma@bookmark}[3][0]{\pdfbookmark[#1]{#2}{#3}}
	\renewcommand{\tma@pageref}[1]{\pageref*{#1}}
\fi

%%%%%%%%%%%%%%%%%%%%%%%%%%%%
%% Question Environment
%%%%%%%%%%%%%%%%%%%%%%%%%%%%

% Set up cref names if cleveref is used
\iftma@usecleveref
  \tma@crefname{question}{question}{questions}
  \tma@crefname{qpart}{part}{parts}
  \tma@crefname{qsubpart}{section}{sections}
\fi

\newenvironment{question}[1][0]{%
	\typeout{Define question environment.}
  \ifthenelse{#1>0}{\setcounter{question}{#1-1}}{\relax}%
  \tma@stepcounter{question}%
  \tma@bookmark{Question \thequestion}{question\thequestion}%
  \makebox[0pt][r]{\large{Q \thequestion.\quad }}%
}{%
  \par \vspace{3em}%
}

\newcommand{\qpart}[1][0]{%
	\typeout{Define qpart command.}
  \ifthenelse{#1>0}{\setcounter{qpart}{#1-1}}{\relax}%
  \tma@stepcounter{qpart}%
  \tma@bookmark[1]{\thequestion.\theqpart}{qpart.\thequestion.\theqpart}%
  \par%
  \makebox[0pt][r]{\large{(\theqpart)\quad }}%
}

\newcommand{\qsubpart}[1][0]{%
	\typeout{Define qsubpart command.}
  \ifthenelse{#1>0}{\setcounter{qsubpart}{#1-1}}{\relax}%
  \tma@stepcounter{qsubpart}%
  \tma@bookmark[2]{\thequestion.\theqpart.\theqsubpart}{qsubpart.\thequestion.\theqpart.\theqsubpart}%
  \par%
  \makebox[0pt][r]{\large{(\theqsubpart)\quad }}%
}

%%%%%%%%%%%%%%%%%%%%%%%%%%%%
%% Mathematical Commands
%%%%%%%%%%%%%%%%%%%%%%%%%%%%

%% Differential Operators
\newcommand{\dd}{\ensuremath{\mathop{}\!\mathrm{d}}}
\newcommand{\e}{\ensuremath{\mathrm{e}}}
\newcommand{\ii}{\ensuremath{\mathrm{i}}}

%% Number Sets
\newcommand{\N}{\ensuremath{\mathbb{N}}}
\newcommand{\Z}{\ensuremath{\mathbb{Z}}}
\newcommand{\Q}{\ensuremath{\mathbb{Q}}}
\newcommand{\R}{\ensuremath{\mathbb{R}}}
\newcommand{\Complex}{\ensuremath{\mathbb{C}}} % Changed from \C to \Complex
\newcommand{\Rr}{\ensuremath{\mathcal{R}}}

%% Vector Notation
\newcommand{\vect}[1]{\ensuremath{\overrightarrow{#1}}} % Changed from \vec to \vect
\newcommand{\ve}[1]{\ensuremath{\bm{#1}}} % Using \bm for bold math symbols

%% Superscript Notations
\newcommand{\st}{\textsuperscript{st}}
\newcommand{\nd}{\textsuperscript{nd}}
\newcommand{\rd}{\textsuperscript{rd}}
\newcommand{\nth}{\textsuperscript{th}}

% Additional Symbols
\newcommand{\rect}{\ensuremath{\sqsubset\!\!\sqsupset}}

%% Combinatorial Notations
\newcommand{\comb}[2]{\ensuremath{{}^{#1}C_{#2}}}
\newcommand{\perm}[2]{\ensuremath{{}^{#1}P_{#2}}}

%% Mathematical Operators
\DeclareMathOperator{\re}{Re}
\DeclareMathOperator{\im}{Im}
\DeclareMathOperator{\Log}{Log}
\DeclareMathOperator{\Arg}{Arg}
\DeclareMathOperator{\Wnd}{Wnd}
\DeclareMathOperator{\Res}{Res}
\DeclareMathOperator{\Ker}{Ker}
\DeclareMathOperator{\Orb}{Orb}
\DeclareMathOperator{\Stab}{Stab}
\DeclareMathOperator{\Fix}{Fix}

%% Derivatives
\newcommand{\deriv}[2]{\frac{\mathrm{d}#1}{\mathrm{d}#2}}
\newcommand{\pderiv}[2]{\frac{\partial #1}{\partial #2}}
\newcommand{\psderiv}[3]{\frac{\partial^2 #1}{\partial #2 \partial #3}}

% Legacy Definitions
\iftma@legacy
	% Redefine \vec to old definition
	\renewcommand{\vec}[1]{\ensuremath{\overrightarrow{#1}}}
	% Redefine \C to old definition
	\providecommand{\C}{\ensuremath{\mathbb{C}}}
	\renewcommand{\C}{\ensuremath{\mathbb{C}}}
\fi

%%%%%%%%%%%%%%%%%%%%%%%%%%%%
%% Theorem Environment
%%%%%%%%%%%%%%%%%%%%%%%%%%%%

\newtheorem{lemma}{Lemma}
\newtheorem{theorem}{Theorem}
%% Define \blacksmiley without loading wasysym
%%\newcommand{\blacksmiley}{\ensuremath{\unicode{263B}}} % Unicode for ☻
\renewcommand{\qedsymbol}{\blacksmiley}

%%%%%%%%%%%%%%%%%%%%%%%%%%%%
%% Miscellaneous Settings
%%%%%%%%%%%%%%%%%%%%%%%%%%%%

\renewcommand{\thefootnote}{\fnsymbol{footnote}}
\numberwithin{equation}{question}
\setlength{\parindent}{0pt}
\setlength{\parskip}{2ex plus 0.3ex minus 0.2ex}

%%%%%%%%%%%%%%%%%%%%%%%%%%%%
%% Header and Footer Settings
%%%%%%%%%%%%%%%%%%%%%%%%%%%%

\pagestyle{fancy}
\fancyhf{} % Clear all headers and footers
\fancyhead[L]{\textrm{\name\ \pin}}
\fancyhead[C]{\textrm{\course\ TMA-\tma}}
\fancyhead[R]{\textrm{Page \thepage\ of \tma@pageref{LastPage}}}
\renewcommand{\headrulewidth}{0pt} % Remove header rule

%%%%%%%%%%%%%%%%%%%%%%%%%%%%
%% End of Package
%%%%%%%%%%%%%%%%%%%%%%%%%%%%

\endinput
