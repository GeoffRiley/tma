\documentclass{ltxdoc}
\PassOptionsToClass{a4paper,11pt,twoside}{article}
\AtBeginDocument{\OnlyDescription}
\AtBeginDocument{\CodelineIndex\EnableCrossrefs}
\AtEndDocument{\PrintIndex}

\makeatletter
\newenvironment{desc}
  {\if@nobreak
    \vskip-\lastskip
    \vspace*{-2.5ex}%
    \fi
    %\decl
    }
  {%\enddecl
  }

\def\opt@font{\normalfont\ttfamily\selectfont}
\newcommand{\opt}[1]{%
    \opt@font{#1}%
    \normalfont\selectfont%
    \index{#1@{\texttt{#1~(option)}}}%
    \index{option!#1@\texttt{#1}}%
}%
\def\env@font{\normalfont\ttfamily\selectfont}
\newcommand{\env}[2][]{%
    \ifblank{#1}{%
        \env@font{#2}%
    }{}%
    \normalfont\selectfont%
    \def\@env{\texttt{#2}}%
    \index{#2@\@env}%
    \index{envionment!#2@\@env}%
}%
\makeatother

\usepackage{ou-tma}
\usepackage{geometry}
\usepackage{tikz}
\usepackage{pgfplots}
\usepgfplotslibrary[units]
\usetikzlibrary{angles,
	quotes,
	calc,
	arrows.meta,
	decorations.markings}
\usepackage{siunitx}
\usepackage{imakeidx}

\pgfplotsset{compat=1.6}

\geometry{margin=1in}
\usepackage[colorlinks=true,linkcolor=blue]{hyperref}
\fancyhf{}
\renewcommand{\headrulewidth}{0.5pt}
\setlength{\headwidth}{\textwidth}
\fancyhead[LE,RO]{\thepage}

\title{The \texttt{ou-tma} Package}
\author{Geoff Riley \\ \href{mailto:geoffr@adaso.com.com}{geoffr@adaso.com}}
\date{Version 1.20, 2025-04-07}

\makeindex[columns=2,intoc,columnseprule,options=-s indexfmt.ist]

\begin{document}

	\maketitle

	\tableofcontents
    \section*{This document\dots}

    \dots is intended as an extra reference to the standard |.dtx| documentation, it will not always be fully in line with the full release but should be updated shortly afterwards.

    It, along with the notes, examples, and tests are not specified or supplied as part of the \textsf{ou-tma} package. Additionally it may contain new concepts that have not yet reached the full release, and may never do so.

	\section{Introduction}

	The \texttt{ou-tma}\index{ou-tma} package provides a set of macros and environments to assist in writing question papers or solutions to Tutor Marked Assessments (TMAs)\index{Tutor Marked Assessment|see{tma}} for Open University courses. It simplifies the formatting of questions, parts, subparts, and includes useful mathematical commands.

	\section{Installation}

	To install the package, place the |ou-tma.sty|\index{ou-tma.sty@{\texttt{ou-tma.sty}}} file in your LaTeX directory or project folder. If using a TeX distribution, you may place it in the local texmf tree.

	\section{Package Options}

	\begin{description}
		\item[\opt{alph}] (default) Question numbering as 1(b)(iii).
		\item[\opt{roman}] Question numbering as 1(ii)(c).
		\item[\opt{cleveref}] Enables automatic referencing with the \texttt{cleveref} package.
		\item[\opt{pdfbookmark}] Adds PDF bookmarks for each question using the \texttt{hyperref} package.
		\item[\opt{legacy}] Enables old definitions of \cs{vec} and \cs{C} for backward compatibility with earlier versions of the package. \emph{Deprecated}.
	\end{description}

	\section{Usage}

	Include\index{inclusion} the package in your LaTeX document:

	\begin{verbatim}
		\documentclass[a4paper,11pt]{article}
		\usepackage[alph,cleveref]{tma}
	\end{verbatim}

	Set your personal information\index{personal information} using the macros \cs{myname}, \cs{mycourse}, \cs{mytma} and \cs{mypin}:

	\begin{verbatim}
		\myname{Your Name}
		\mycourse{Course Code}
		\mytma{TMA Number}
		\mypin{Your PIN}
	\end{verbatim}

	If you wish to set a specific date for your report, use \cs{setdate}:

	\begin{verbatim}
		\setdate{21 November 2024}
	\end{verbatim}

	This is a free format field, any rendition of `date' may be given. You can retrieve the report date later using the \cs{thedate} command.

	\section{Commands and Environments}

	\subsection{Question Environment}

	\PrintEnvName{question} Begin answering a question with the \env{question} environment:

	\begin{verbatim}
	\begin{question}
		% Question text
	\end{question}
\end{verbatim}

Questions are automatically numbered from 1 unless an integer parameter is given in square brackets after the opening of the environment, for example:

\begin{verbatim}
\begin{question}[6] % set the question number to be 6
	% Question text
\end{question}
\end{verbatim}

\subsection{Parts and Subparts}

Within a question environment, divisions are achieved using \verb|\qpart| and \verb|\qsubpart| to create parts and subparts:

\begin{verbatim}
\qpart % For parts
\qsubpart % For subparts
\end{verbatim}

Again, the numbering for parts and subparts begin at 1 unless otherwise prescribed by an integer within square brackets after the appropriate command.

\subsection{Mathematical Commands}

The package provides several mathematical commands:

\begin{itemize}
\item \verb|\N|, \verb|\Z|, \verb|\Q|, \verb|\R|, \verb|\Complex|: Number sets.
\item \verb|\vect{v}|: Vector notation with an arrow over the variable.
\item \verb|\ve{v}|: Bold vector notation.
\item \verb|\dd|, \verb|\e|, \verb|\ii|: Differential operator, Euler's number, imaginary unit.
\item Superscript ordinals: \verb|\st|, \verb|\nd|, \verb|\rd|, \verb|\nth|.
\end{itemize}

\section{Examples}

\subsection{Sample Question}

\begin{verbatim}
\begin{question}[4]
  Prove that the set \N{} of natural numbers is infinite.

  \qpart
    What does it mean for a set to be infinite?

  \qpart
    Assume, for contradiction the \N{} is a finite set.
    Under this assumption, what can you say about the
    number of elements in \N{} and the existence of a
    largest natural number?

  \qpart
    Explain how the successor function $s(n) = n + 1$
    leads to a contradiction when applied to the largest
    natural number assumption of part (b).

  \qpart
    Based on the contradiction found in part (c), conclude
    whether the set \N{} is finite or infinite, and briefly
    explain why.
\end{question}
\end{verbatim}

Note that the default part numbering is in the form (a), (b), (c)\dots If the option \verb|[roman]| is given when loading the \verb|tma| package, then the numbering will be (i), (ii), (iii)\dots instead.

\begin{center}
\begin{minipage}{0.7\textwidth}
\begin{question}[4]
  Prove that the set \N{} of natural numbers is infinite.

  \qpart
    What does it mean for a set to be infinite?

  \qpart
    Assume, for contradiction the \N{} is a finite set.
    Under this assumption, what can you say about the
    number of elements in \N{} and the existence of a
    largest natural number?

  \qpart
    Explain how the successor function $s(n) = n + 1$
    leads to a contradiction when applied to the largest
    natural number assumption of part (b).

  \qpart
    Based on the contradiction found in part (c), conclude
    whether the set \N{} is finite or infinite, and briefly
    explain why.
\end{question}
\end{minipage}
\end{center}

\subsection{Using subsections}

\begin{verbatim}
  \setquestionstring{Question}
  \begin{question}[5]
    Find the roots of $x^2 +5x +6 = 0$
    \qpart %
    \qsubpart %
    The quadratic equation is \[
    x=\frac{-b\pm\sqrt{b^2-4ac}}{2a}
    \]
    \qsubpart %
    When $a=1$, $b=5$ and $c=6$:
    \begin{align}
      x &= \frac{-(5)\pm\sqrt{(5)^2-4\times1\times6}}{2\times1} \\
        &= \frac{-5\pm\sqrt{25-24}}{2}
        &= \frac{-5\pm\sqrt{1}}{2}
    \end{align}
    \qsubpart %
    So, $x=\frac{-5+1}{2}$ or $x=\frac{-5-1}{2}$.\\[0.5cm]

    Therefore the roots of $x^2 +5x +6 = 0$ are $x=-2$ or $x=-3$.
  \end{question}
\end{verbatim}

\begin{center}
	\begin{minipage}{0.7\textwidth}
		\setquestionstring{Question}
		\begin{question}[5]
			Find the roots of $x^2 +5x +6 = 0$
			\qpart %
			\qsubpart %
			The quadratic equation is \[
			x=\frac{-b\pm\sqrt{b^2-4ac}}{2a}
			\]
			\qsubpart %
			When $a=1$, $b=5$ and $c=6$:
			\begin{align}
				x &= \frac{-(5)\pm\sqrt{(5)^2-4\times1\times6}}{2\times1} \\
				&= \dfrac{-5\pm\sqrt{25-24}}{2} \\
				&= \dfrac{-5\pm\sqrt{1}}{2}
			\end{align}
			\qsubpart %
			So, $x=\frac{-5+1}{2}=-2$ or $x=\frac{-5-1}{2}=-3$.\\[0.5cm]

			Therefore the roots of $x^2 +5x +6 = 0$ are $x=-2$ or $x=-3$.
		\end{question}
	\end{minipage}
\end{center}

\subsection{Using Mathematical Commands}

\begin{verbatim}
Relationship between number sets:
\begin{itemize}
  \item \N{} (Natural numbers) $\subseteq \Z$ (Integers);
  every natural number is also an integer.
  \item \Z{} (Integers) $\subseteq \Q$ (Rational numbers);
  every integer is also a rational number.
  \item \Q{} (Rational numbers) $\subseteq \R$ (Real numbers);
  every rational number is also a real number.
  \item \Complex{} (Complex numbers) $\supseteq \R$ (Real number);
  complex numbers include real numbers as a subset, since they
  can be represented by $a+\ii b$ where $a$ and $b$ are real numbers.
\end{itemize}
Hence, $\N\subseteq\Z\subseteq\Q\subseteq\R\subseteq\Complex$.\\[0.5cm]

Euler's formula states that $\e^{\ii\theta} = \cos\theta + \ii\sin\theta$

A combination is a selection of objects without regard to order.
\begin{align}
      	C(n,r) &= \comb{n}{r}\\
      	&= \frac{n!}{r!(n-r)!}
\end{align}
Where $n$ is the total number of objects, $r$ is the number of objects to
be selected and $!$ denotes the factorial
(e.g., $5! = 5\times4\times3\times2\times1$).

A permutation is an arrangement of objects in a specific order.
\begin{align}
      	P(n.r) &= \perm{n}{r}\\
      	&= \frac{n!}{(n-r)!}
\end{align}
Where $n$ is the total number of objects, $r$ is the number of objects to
be selected and $!$ denotes the factorial.
\end{verbatim}

\begin{center}
    \begin{minipage}{0.7\textwidth}
        Relationship between number sets:
        \begin{itemize}
          \item \N{} (Natural numbers) $\subseteq \Z$ (Integers); every natural number is also an integer.
          \item \Z{} (Integers) $\subseteq \Q$ (Rational numbers); every integer is also a rational number.
          \item \Q{} (Rational numbers) $\subseteq \R$ (Real numbers); every rational number is also a real number.
          \item \Complex{} (Complex numbers) $\supseteq \R$ (Real number); complex numbers include real numbers as a subset, since they can be represented by $a+\ii b$ where $a$ and $b$ are real numbers.
        \end{itemize}
        Hence, $\N\subseteq\Z\subseteq\Q\subseteq\R\subseteq\Complex$.\\[0.5cm]

        Euler's formula states that $\e^{\ii\theta} = \cos\theta + \ii\sin\theta$\\[0.5cm]

        A combination is a selection of objects without regard to order.
    	\begin{align}
        	C(n,r) &= \comb{n}{r}\\
        	&= \frac{n!}{r!(n-r)!}
        \end{align}
        Where $n$ is the total number of objects, $r$ is the number of objects to be selected and $!$ denotes the factorial (e.g., $5! = 5\times4\times3\times2\times1$).

        A permutation is an arrangement of objects in a specific order.
        \begin{align}
        	P(n,r) &= \perm{n}{r}\\
        	&= \frac{n!}{(n-r)!}
        \end{align}
        Where $n$ is the total number of objects, $r$ is the number of objects to be selected and $!$ denotes the factorial.
    \end{minipage}
\end{center}

\section{Vectors}
\begin{verbatim}
\begin{question}[6]
Figure~\ref{fig:q6_vectors} shows the vectors involved in this problem.
\qpart % (a)
The given angles are bearings, from the compass North with a clockwise
positive reading.
\begin{align*}
	|\ve{b}| &= 120\qquad|\ve{w}| = 30\\
	\intertext{The vector \ve{b} in component form is made using the
		acute angle in the lower quadrant, i.e. $\ang{180}-\ang{120}=\ang{60}$,
		this implies that the \ve{j}-component is negative:}
	\ve{b} &= 170 \sin(\ang{60})\ve{i} - 170 \cos(\ang{60})\ve{j}\\
	&=\frac{170\sqrt{3}}{2}\ve{i} - \frac{170}{2}\ve{j}\\
	\vdots
\end{align*}
Therefore, the two vectors in component form are:\[
\ve{b}=(147.22\,\ve{i} - 85.00\,\ve{j})\,\unit{\km\per\hour}
\] and \[
\ve{w}=(21.21\,\ve{i} + 21.21\,\ve{j})\,\unit{\km\per\hour}
\] (all to 2 d.p.)
\end{question}
\end{verbatim}
\begin{figure}
	\centering
	\begin{tikzpicture}[
		>=Latex,
		]
		% Vector setup
\coordinate (ij) at (10,2) ;
\node (vi) at ($(ij)+(1,0)$) {$\ve{i}$};
\node (vj) at ($(ij)+(0,1)$) {$\ve{j}$};
%
\coordinate (vecbase) at (3,1.5);
\coordinate (vecnorth) at ($(vecbase)+(0,1)$);
\node (vw) at (4,2.5) {$\ve{w}$};
\node (vb) at (4,0) {$\ve{b}$};
% Compass setup
\coordinate (Compass) at (7,2);
\node (CN) at ($(Compass)+(0,1)$) {$N$};
\coordinate (CS) at ($(Compass)-(0,0.5)$);
\coordinate (CW) at ($(Compass)-(0.5,0)$);
\coordinate (CE) at ($(Compass)+(0.5,0)$);
% Drawing
\begin{scope}[very thick, decoration={markings,mark=at position 0.8 with \arrow{>}}]
	\draw[postaction={decorate}] (ij)--(vi);
	\draw[postaction={decorate}] (ij)--(vj);
	\draw[postaction={decorate}] (vecbase)--(vw) ;
	\draw[postaction={decorate}] (vecbase)--(vb) ;
\end{scope}
\draw[dotted] ($(vecbase)-(0,2)$)--($(vecbase)+(0,2)$) ;
\draw[-Latex,thick] (CS) -- (CN);
\draw[thick] (CW) -- (CE);
\draw (Compass) circle [radius=0.25];
\pic [draw, angle radius=0.5cm, angle eccentricity=2, "$\ang{45}$"] {angle=vw--vecbase--vecnorth};
\pic [draw, angle radius=0.4cm, angle eccentricity=2.5, "$\ang{120}$"] {angle=vb--vecbase--vecnorth};
% Key
\node[draw,text width=6cm,align=justify] at (8,0.5) {$\ve{w}=\qty{30}{\km\per\hour}$ from south-west\\$\ve{b}=\qty{170}{\km\per\hour}$ bearing \ang{120}};
	\end{tikzpicture}
	\caption{Vectors}
	\label{fig:q6_vectors}
\end{figure}
\begin{center}
\begin{minipage}{0.7\textwidth}
\begin{question}[6]
Figure~\ref{fig:q6_vectors} shows the vectors involved in this problem.
\qpart % (a)
The given angles are bearings, from the compass North with a clockwise
positive reading.
\begin{align*}
	|\ve{b}| &= 120\qquad|\ve{w}| = 30\\
	\intertext{The vector \ve{b} in component form is made using the
		acute angle in the lower quadrant, i.e. $\ang{180}-\ang{120}=\ang{60}$,
		this implies that the \ve{j}-component is negative:}
	\ve{b} &= 170 \sin(\ang{60})\ve{i} - 170 \cos(\ang{60})\ve{j}\\
	&=\frac{170\sqrt{3}}{2}\ve{i} - \frac{170}{2}\ve{j}\\
	\vdots
\end{align*}
Therefore, the two vectors in component form are:\[
\ve{b}=(147.22\,\ve{i} - 85.00\,\ve{j})\,\unit{\km\per\hour}
\] and \[
\ve{w}=(21.21\,\ve{i} + 21.21\,\ve{j})\,\unit{\km\per\hour}
\] (all to 2 d.p.)
\end{question}
\end{minipage}
\end{center}

Note, the command \verb|\ang{}| is found in the package  \href{https://ctan.org/pkg/siunitx}{\texttt{siunitx}}.

\section{Command summary}

\begin{desc}
	\verb|\st|\\
	\verb|\nd|\\
	\verb|\rd|\\
	\verb|\nth|
\end{desc}
Commands to set ordinal indicators after numbers. For example, 1\st, 2\nd, 3\rd and 4\nth.

\section{Change Log}

\begin{itemize}
\item \textbf{v1.20 (2025-04-07)}: Package name changed from `tma' to `ou-tma' to become a little more descriptive and to abide by the minimum package name length suggested by CTAN.
\item \textbf{v1.19 (2025-02-18)}: PDF metadata (apparently) solved with help from Steve Mayers; all down to the use of commands being used as string containers. New (\LaTeX3) commands are robust and fail to expand within the context of the metadata and bookmarks, old (\LaTeX2e) commands are fragile and correctly expanded. I have a mix of old commands and new variables now.
\item \textbf{v1.18 (2025-02-16)}: PDF metadata doesn't set correctly so I have removed it: the cause is an incompatibility between \LaTeX\ unicode and the PDF restricted character allowance.
\item \textbf{v1.17 (2025-02-13)}: Rewritten with \LaTeX-3 syntax from the `xparse' package to make commands less fragile. Finally got alignment of part and subpart numbering to line up correctly.
\item \textbf{v1.16 (2024-11-22)}: Added File, Properties to pdf files using the hyperref setup system when in pdfbookmark mode.
\item \textbf{v1.15 (2024-11-21)}: Added \verb|\setdate{}| and \verb|\thedate| for setting and using a fixed report date.
\item \textbf{v1.14 (2024-11-17)}: Allow setting of question string with \verb|\setquestionstring{}|; set indentation for \verb|\qpart| and \verb|\qsubpart| from Steve Mayers contribution.
\item \textbf{v1.13 (2024-11-16)}: Attempt at placing part and subpart marks on the same line when there is no intervening text.
\item \textbf{v1.12 (2024-11-08)}: Fixed error with \verb|\renewcommand{\C}|; changed to \\ \verb|\providecommand{\C}|.
\item \textbf{v1.11 (2024-11-08)}: Added 'legacy' option to allow old definitions of \verb|\vec| and \verb|\C|.
\item \textbf{v1.10 (2024-09-10)}: Ensured commands are correctly defined before being redefined!
\item \textbf{v1.09 (2024-09-10)}: Improved code readability.
\item \textbf{v1.08 (2024-09-05)}: Consolidated geometry settings and adjusted package loading order.
\item \textbf{v1.07 (2024-09-04)}: Avoided redefining standard \LaTeX\ commands to avoid clashes. Fixed error with \verb|\ExecuteOptions| in package file. Implemented flags to handle the various options simpler.
\item \textbf{v1.06 (2024-08-28)}: Updated definitions of \verb|\comb|, \verb|\perm| to correctly work after a superscript.
\item \textbf{v1.05 (2024-08-28)}: Updated definitions of \verb|\ve|, \verb|\vec| to allow use in both maths and text mode.
\item \textbf{v1.04 (2024-08-26)}: Updated usage of 'fancyhdr' to remove deprecated commands.
\end{itemize}

\section{License}

This work may be distributed and/or modified under the conditions of the LaTeX Project Public License (LPPL), version 1.3c or later.

\section{Acknowledgments}

Special thanks to Peter McFarlane, Bob Margolis, Rob Lynas, Steve Mayer, Tim Dale and to all contributors and, especially, the Open University community.

\end{document}
