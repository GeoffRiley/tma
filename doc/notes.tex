\documentclass[a4paper,12pt]{article}
\usepackage[pdfbookmark]{tma}
\usepackage{multirow}
\usepackage{hologo}

\myname{Peter McFarlane {\scriptsize with additions by} Geoff Riley}
\mypin{A1234567}
\mycourse{\LaTeX 101}
\mytma{notes}

\fancyhead[L]{tma.sty notes}
\fancyhead[C]{}

\newcommand{\MiKTeX}{\hologo{MiKTeX}}

\begin{document}
	\maketitle
	
	\hfill\fbox{Update for 2024/11/08 v1.04}
	
	\section*{Notes}
	\thispagestyle{empty}
	
	This document describes Peter McFarlane's \verb|tma| package.  To use it, the \verb|tma.sty| file should be in the 
	same directory/folder as your main \verb|.tex| source file or in the file path for your compiler. Using \MiKTeX\footnote{The latest version of \MiKTeX\ is always downloadable from \url{https://miktex.org/}.} v24.1
	on Windows, installed for all users, this would be:\\
	\verb|C:\Program Files\MiKTex\tex\latex\tma|\\
	(you need to create the \verb|tma| folder).
	
	Once you have added \verb|tma.sty| to the directory, it is necessary to let \MiKTeX\ know that it is there. So
	load \MiKTeX, go to \MiKTeX\ Settings and press the `Refresh FNDB' (\textbf{F}ile \textbf{N}ame \textbf{D}ata\textbf{B}ase) button that tells \LaTeX\ to remake the
	database of all the files, then it will find \verb|tma.sty|. See:\\
	\url{http://docs.miktex.org/manual/configuring.html#fndbupdate}.
	
	Other \TeX\ installations will have similar centralised file storage settings for style files.
	
	The contents of the file in which you write a TMA should look like this:
	\begin{verbatim}
		\documentclass[a4paper,12pt]{article}
		\usepackage{tma}
		%% or include options [roman] or [alph]:
		% \usepackage[roman]{tma} % For Roman numerals in subparts
		% \usepackage[alph]{tma}  % For the default alphabetical letters in subparts
		
		\myname{Nosmo King}
		\mypin{A1234567}
		\mycourse{L101}
		\mytma{01}
		
		\begin{document}
			
			\include{question_01}
			\include{question_02}
			
		\end{document}
	\end{verbatim}
	
	Using \verb|\include| and writing your questions in separate files is unnecessary.  But do start each
	question with \verb|\begin{question}| and end it with \verb|\end{question}|, whether or not you use
	separate files for each question, or write all the questions in the main text.
	
	If you wish to have margin notes, then include \verb|\marginnotes| in your preamble, whereupon
	\verb|\marginnote{}| is equivalent to \verb|\marginpar{}| and places the context of the brackets in the margin.
	
	The question numbers and question parts appear in the margin.
	
	The normal sequence for numbering of questions is Arabic numerals for the main question numbers,
	letters for	the parts, and Roman numerals for the subparts. If you are on a module that uses Roman
	numerals for the part, such as M381, then you can pass the option \verb|[roman]| to the
	\verb|\usepackage| command to vary the numbering system. See below for further details of options.
	
	If you want to skip a question (i.e. jump straight from question 1 to question 3, then use (for example)
	\verb|\begin{question}[3]|. To get parts of questions (a), (b), (c) etc, use \verb|\qpart|. To skip part
	question numbers \verb|\qpart[3]| would force a (c).  For subparts (i), (ii), (iii), etc, then use \verb|\qsubpart|.
	
	There is a slight gap between paragraphs and no indent, although, as mentioned, the
	question numbers are in the left-hand margin.
	
	\subsection*{Options}
	
	When requesting the \verb|tma| package with the \verb|\usepackage| command, it is possible to pass one or 
	more optional parameters to influence how the package will operate.  Just as it is typical to let the 
	\verb|\documentclass| have options specifying the paper and font size, so can many other packages being given options.
	
	\begin{table}[h]
		\centering
		\begin{tabular}[2]{lp{0.8\textwidth}}
			\hline
			\noalign{\vskip 3pt}
			\textbf{\sc Option}      & \textbf{\sc Effect} \\
			\hline
			\noalign{\vskip 3pt}
			\verb|[alph]|        & (default) question numbering as 1(b)(iii) \\
			\verb|[roman]|       & varies question numbering to sequence used by M381 i.e. 1(ii)(c) \\
			\verb|[cleveref]|    & question numbering creates automatic referencing for use with \verb|cleveref| package \\
			\verb|[pdfbookmark]| & add pdf bookmarks for each question using \verb|hyperref| package \\
			\verb|[legacy]|      & add backward compatibility for old versions of \verb|tma| package \\
			\hline
		\end{tabular}
		\caption{Options available for \texttt{tma.sty}}
		\label{tab:options}
	\end{table}
	
	To use a package option, place the option(s) before the package name in square brackets, for example:\\
	\verb|\usepackage[roman,cleveref]{tma}|
	
	
	\subsection*{New commands}
	
	New commands provided by the package include the following:
	
	\hfil\verb|\R|\qquad\R\hfil\verb|\N|\qquad\N\hfil\verb|\Z|\qquad\Z\hfil\verb|\Q|\qquad\Q\hfil\verb|\Complex|\qquad\Complex
	
	In typeset mathematics, constants such as \e , \ii \ $(\sqrt{-1})$, and $\uppi$ should be not be italic,
	nor should \dd \ (as in $\deriv{y}{x}$ or $\int \e^x \,\dd x$). Hence:
	
	\hfil\verb|\dd|\qquad\dd\hfil\verb|\e|\qquad\e\hfil\verb|\ii|\qquad\ii\hfil\verb|\uppi|\qquad$\uppi$
	
	(\verb|\d| produces a dot over the following character. \verb|\i| produces a dotless i to enable accents
	over a na\"\i ve \i. \verb|\uppi| is provided by the upgreek package (and can be used for all Greek
	letters).  \verb|\dd| also adds a small space before the $\dd x$ so that it is slightly separated from the
	integral equation instead of being part of it.
	
	\qquad \qquad \verb|\deriv{y}{x}|\qquad  $\genfrac{}{}{}{0}{\dd y}{\dd x}$	%
	\qquad \qquad \verb|\pderiv{y}{x}|\qquad  $\genfrac{}{}{}{0}{\partial y}{\partial x}$
	
	\qquad \qquad \verb|\psderiv{z}{x}{y}|\qquad  $\genfrac{}{}{}{0}{\partial ^2z}{\partial y\partial x}$ %
	
	Other commands, some of which have been plagiarised from other peoples' style files, include mathematical
	functions for the principle logarithm and various group theory and complex analysis functions.  Also, a
	\verb|\rect| is included for M208 people (other shapes are included by virtue of the \verb|wasysym| package).
	Commands available are given in table~\ref{tab:commands}; the upper set of commands will work in text or maths mode, whilst the lower commands are only designed to work in maths mode.
	
	\subsection*{Backward compatibility}
	
	As of version 1.04 of the \verb|tma| package, two commands have been renamed to avoid clashes with other libraries. \verb|\C| has been renamed \verb|\Complex|, and \verb|\vec| has been renamed \verb|\vect|. In order to allow order documents to still use the new version of the package, an additional option has been provided, \verb|[legacy]|, and this will reimplement the old names.  These old names, however, are now deprecated and may be removed in future issues.
	
	\begin{table}[t]
		\centering
		\begin{tabular}[3]{lll}
			\hline
			\noalign{\vskip 3pt}
			\textbf{\sc Command} & \textbf{\sc Example} & \textbf{\sc Note} \\
			\hline
			\noalign{\vskip 3pt}
			\verb|\Rr|         & \Rr         & (for a region)                                \\
			\verb|\ve{j}|      & $\ve{j}$    & for emboldened vectors                        \\
			\verb|\vect{AB}|   & $\vect{AB}$ & for traditional vectors                       \\
			\verb|1\st|        & $1\st$      & also \verb|\nd|, \verb|\rd|, \verb|\nth|      \\
			\verb|\rect|       & \rect       &  \\
			\verb|\comb{3}{5}| & \comb{5}{3} &  \\
			\verb|\perm{3}{5}| & \perm{5}{0} &  \\[4pt]
			\hline
			\noalign{\vskip 4pt}
			\verb|\re|         & $\re $      & \verb|\Re| will produce the traditional $\Re$ \\
			\verb|\im|         & $\im $      & \verb|\Im| will produce $\Im$                 \\
			\verb|\Log|        & $\Log $     &  \\
			\verb|\Arg|        & $\Arg $     &  \\
			\verb|\Wnd|        & $\Wnd $     &  \\
			\verb|\Res|        & $\Res $     &  \\
			\verb|\Ker|        & $\Ker $     &  \\
			\verb|\Res|        & $\Res $     &  \\
			\verb|\Orb|        & $\Orb $     &  \\
			\verb|\Stab|       & $\Stab $    &  \\
			\verb|\Fix|        & $\Fix $     &  \\ \hline
		\end{tabular}
		\caption{Mathematical commands that are made available with \texttt{tma.sty}}
		\label{tab:commands}
	\end{table}
	
	\newpage
	\subsection*{Packages automatically loaded}
	
	Some standard packages are automatically loaded when the \verb|tma| package is used. These, in turn, load other packages. A summary of the packages so loaded is listed in table~\ref{tab:packages}.
	
	\begin{table}[t]
		\centering
		\begin{tabular}{cl}
			\hline
			\noalign{\vskip 3pt}
			\textbf{\sc Package}   & \textbf{\sc Notes} \\
			\hline
			\noalign{\vskip 3pt}
			\texttt{amssymb}   & {Also loads \texttt{amsfonts}} \\
			\texttt{amsmath}   & {Also loads \texttt{amstext}, \texttt{amsgen}, \texttt{amsbsy} and \texttt{amsopn}} \\
			\texttt{amsthm}    & \\
			\texttt{upgreek}   & \\
			\texttt{wasysym}   & \\
			\texttt{bm}        &  \begin{minipage}[t]{0.8\textwidth}
				This allows you to embolden maths formulae: \\
				\verb|$\int \e^x \dd x=\bm{\int \e^x \dd x}$| \\
				$$\int \e^x \dd x=\bm{\int \e^x \dd x}$$
			\end{minipage}\\
			\texttt{fancyhdr}  & \\
			\texttt{geometry}  & {Also loads \texttt{keyval}, \texttt{ifvtex} and \texttt{iftex}} \\
			\texttt{xifthen}   & {Also loads \texttt{calc}, \texttt{ifthen} and \texttt{ifmtarg}} \\
			\texttt{verbatim}  & \\
			\texttt{graphicx}  & {Also loads \texttt{graphics} and \texttt{trig}} \\
			\texttt{lastpage}  & {Also loads \texttt{lastpage2e} and \texttt{lastpagemodern}} \\[4pt]
			\hline
			\noalign{\vskip 4pt}
			\texttt{cleveref}  & {This is only loaded if the \verb|cleveref| option is given to the style.} \\
			\texttt{hyperref}  & {This is only loaded if the \verb|cleveref| or \verb|pdfbookmark| option is given to the style.} \\
			\hline
		\end{tabular}
		\caption{Packages auto-loaded by \texttt{tma.sty}}
		\label{tab:packages}
	\end{table}
	
	Note that the latter two packages are conditionally loaded only when the appropriate option is specified.
	
	\subsection*{\dots and finally}
	
	Any comments, ideas, or suggestions (either of style or for more macros) are welcomed.
	
	\vskip 1in
	
	$$\dots\quad\infty\quad\maltese\quad\infty\quad\dots$$
	
\end{document}

